\documentclass[12pt,a4paper]{report}

% Essential packages
\usepackage[utf8]{inputenc}
\usepackage[T1]{fontenc}
\usepackage[french]{babel}
\usepackage{graphicx}
\usepackage{geometry}
\usepackage{fancyhdr}
\usepackage{hyperref}
\usepackage{color}
\usepackage{titlesec}
\usepackage{enumitem}

% Page geometry
\geometry{
    a4paper,
    top=2.5cm,
    bottom=2.5cm,
    left=2.5cm,
    right=2.5cm
}

% Header and footer configuration
\pagestyle{fancy}
\fancyhf{}
\rhead{\thepage}
\lhead{Rapport de Stage}

% Title page configuration
\title{
    \huge\textbf{Rapport de Stage}\\
    \vspace{1cm}
    \Large{Nom de l'entreprise}\\
    \vspace{0.5cm}
    \large{Période du stage: XX/XX/XXXX au XX/XX/XXXX}
}

\author{
    \Large{Prénom NOM}\\
    \vspace{0.3cm}
    \large{Formation / Niveau d'études}\\
    \vspace{0.3cm}
    \large{Établissement}
}

\date{\today}

\begin{document}

\maketitle

% Abstract / Résumé
\begin{abstract}
    \noindent Ce rapport présente le stage effectué au sein de [Nom de l'entreprise] durant [période]. 
    Il détaille les missions réalisées, les compétences acquises ainsi que les objectifs atteints 
    pendant cette période de formation professionnelle.
\end{abstract}

% Table of contents
\tableofcontents
\newpage

% Remerciements
\chapter*{Remerciements}
Je tiens à remercier...
\newpage

% Introduction
\chapter{Introduction}
\section{Contexte du stage}
\section{Présentation de l'entreprise}
\section{Objectifs du stage}

% Corps du rapport
\chapter{Présentation de l'entreprise}
\section{Histoire et activités}
\section{Organisation}
\section{Secteur d'activité}

\chapter{Missions et tâches réalisées}
\section{Mission principale}
\section{Tâches quotidiennes}
\section{Projets spécifiques}

\chapter{Compétences acquises}
\section{Compétences techniques}
\section{Compétences transversales}
\section{Savoir-être}

\chapter{Analyse et réflexion}
\section{Difficultés rencontrées}
\section{Solutions apportées}
\section{Bilan personnel}

% Conclusion
\chapter{Conclusion}
\section{Bilan du stage}
\section{Perspectives}

% Bibliographie
\bibliographystyle{plain}
\bibliography{references}

% Annexes
\appendix
\chapter{Annexes}
\section{Documents complémentaires}
\section{Glossaire}

\end{document} 