\section{Bilan du Stage}
Ce stage de six mois chez Carrefour a été une expérience enrichissante tant sur le plan technique que professionnel. J'ai pu participer à un projet d'envergure qui a permis de moderniser la plateforme e-commerce de l'entreprise. Les objectifs fixés au début du stage ont été atteints, avec des résultats concrets en termes de performance et d'expérience utilisateur.

\section{Apports Personnels}
Ce stage m'a permis de :
\begin{itemize}
    \item Acquérir une solide expérience en développement web moderne
    \item Maîtriser les architectures microservices
    \item Développer mes compétences en DevOps
    \item Améliorer mes capacités de travail en équipe
    \item Renforcer mes compétences en gestion de projet agile
\end{itemize}

\section{Perspectives d'Évolution}
Les perspectives d'évolution suite à ce stage sont multiples :
\begin{itemize}
    \item Poursuite du développement de la plateforme e-commerce
    \item Intégration de nouvelles fonctionnalités basées sur l'IA
    \item Optimisation continue des performances
    \item Expansion internationale de la plateforme
    \item Amélioration de l'expérience mobile
\end{itemize}

\section{Remerciements}
Je tiens à remercier :
\begin{itemize}
    \item M. Jean Dupont, mon tuteur en entreprise, pour son accompagnement et ses précieux conseils
    \item Mme Marie Martin, ma tutrice universitaire, pour son suivi et son soutien
    \item Toute l'équipe de développement e-commerce pour leur accueil et leur collaboration
    \item Les équipes DevOps pour leur assistance technique
    \item La direction de Carrefour pour cette opportunité de stage
\end{itemize} 