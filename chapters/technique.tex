\section{Technologies Utilisées}
\subsection{Environnement de Développement}
L'environnement de développement était basé sur :
\begin{itemize}
    \item Système d'exploitation : Linux Ubuntu
    \item IDE : Visual Studio Code
    \item Version Control : Git avec GitHub Enterprise
    \item CI/CD : Jenkins
    \item Containerisation : Docker et Kubernetes
\end{itemize}

\subsection{Outils et Technologies}
\begin{itemize}
    \item Frontend :
        \begin{itemize}
            \item React.js avec TypeScript
            \item Next.js pour le SSR
            \item Tailwind CSS pour le styling
            \item Redux pour la gestion d'état
            \item React Query pour la gestion des données
            \item Formik pour la gestion des formulaires
            \item React Hook Form pour la validation
            \item React Table pour les tableaux de données
            \item Chart.js pour les visualisations
        \end{itemize}
    \item Backend :
        \begin{itemize}
            \item Node.js avec Express
            \item MongoDB pour la base de données
            \item Redis pour le cache
            \item RabbitMQ pour la messagerie
            \item Elasticsearch pour la recherche
            \item Stripe pour les paiements
            \item SendGrid pour les emails
            \item AWS S3 pour le stockage des images
        \end{itemize}
    \item DevOps :
        \begin{itemize}
            \item AWS pour l'infrastructure cloud
            \item Terraform pour l'IaC
            \item Prometheus et Grafana pour le monitoring
            \item ELK Stack pour la gestion des logs
            \item SonarQube pour l'analyse de code
        \end{itemize}
\end{itemize}

\section{Méthodologie de Travail}
La méthodologie de travail suivie était basée sur Scrum :
\begin{itemize}
    \item Sprints de 2 semaines
    \item Daily stand-up meetings
    \item Revue de sprint et rétrospective
    \item Pair programming pour les tâches complexes
    \item Code review systématique
    \item Tests automatisés (unitaires, d'intégration, de performance)
    \item Documentation continue
\end{itemize}

\section{Difficultés Rencontrées}
Les principales difficultés techniques rencontrées étaient :
\begin{itemize}
    \item Migration progressive vers la nouvelle architecture
    \item Gestion de la cohérence des données entre microservices
    \item Optimisation des performances sur mobile
    \item Intégration des systèmes de paiement
    \item Mise en place du monitoring distribué
    \item Gestion des stocks en temps réel
    \item Synchronisation des catalogues produits
    \item Optimisation des recherches avancées
    \item Sécurisation des transactions
    \item Gestion des pics de charge
\end{itemize}

\section{Apports Techniques}
Les principaux apports techniques du stage sont :
\begin{itemize}
    \item Mise en place d'une architecture microservices robuste
    \item Implémentation d'un système de cache efficace
    \item Développement d'une PWA performante
    \item Automatisation des tests et du déploiement
    \item Documentation technique complète
    \item Système de recherche avancée optimisé
    \item Gestion des stocks en temps réel
    \item Système de paiement sécurisé
    \item Tableaux de bord analytiques
    \item Système de notification en temps réel
\end{itemize} 