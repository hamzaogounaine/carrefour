\section{Historique de l'Entreprise}
Carrefour est un groupe international de distribution fondé en 1959 en France. Depuis sa création, l'entreprise n'a cessé de se développer pour devenir l'un des leaders mondiaux de la grande distribution. En 2024, Carrefour compte plus de 12 000 magasins dans plus de 30 pays, employant plus de 320 000 collaborateurs.

\section{Activités de l'Entreprise}
\subsection{Secteur d'Activité}
Carrefour opère dans le secteur de la grande distribution, proposant une large gamme de produits alimentaires et non-alimentaires. L'entreprise est présente dans plusieurs formats de magasins : hypermarchés, supermarchés, magasins de proximité et commerce en ligne.

\subsection{Produits et Services}
Carrefour propose :
\begin{itemize}
    \item Une large gamme de produits alimentaires
    \item Des produits de grande consommation
    \item Des services bancaires et d'assurance
    \item Une plateforme e-commerce complète
    \item Des services de livraison et de drive
\end{itemize}

\section{Organisation de l'Entreprise}
\subsection{Structure Organisationnelle}
Carrefour est organisé en plusieurs divisions :
\begin{itemize}
    \item Direction Générale
    \item Division E-commerce
    \item Division Marketing
    \item Division Logistique
    \item Division IT et Digital
\end{itemize}

\subsection{Service d'Accueil}
Le stage s'est déroulé au sein de la Division IT et Digital, plus précisément dans l'équipe de développement e-commerce. Cette équipe est responsable du développement et de la maintenance de la plateforme de vente en ligne de Carrefour.

\section{Environnement de Travail}
L'environnement de travail est moderne et collaboratif, avec :
\begin{itemize}
    \item Des espaces de travail ouverts et collaboratifs
    \item Des outils de développement modernes (Git, Jira, Confluence)
    \item Une infrastructure cloud performante
    \item Des méthodes de travail agiles (Scrum)
    \item Des équipements de développement de pointe
\end{itemize} 